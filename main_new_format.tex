\documentclass[10pt]{article}

\usepackage{hicss51}
\usepackage{times}
\usepackage[none]{hyphenat}
\usepackage{url}
\usepackage{latexsym}
\usepackage{indentfirst}
\usepackage{graphicx}

\newcommand{\sansserifformat}[1]{\fontfamily{cmss}{ #1}}%

\let\proof\relax\let\endproof\relax\usepackage{amsthm}
\let\labelindent\relax\usepackage{enumitem}% NEEDED TO MAKE HYPERREF LIKE REFERENCES
\makeatletter
\let\NAT@parse\undefined
\makeatother
\usepackage{ragged2e}
\usepackage[bookmarks=false,colorlinks,urlcolor=blue,%
linkcolor=magenta,citecolor=red,linktocpage=true,breaklinks=true]{hyperref}
\usepackage{xspace}
\usepackage{graphics}
\usepackage{inconsolata}
\usepackage{cleveref}
\usepackage{xcolor}
\usepackage{fancyvrb}
\usepackage{listings}
\usepackage{xparse}
\usepackage{tikz}
\usetikzlibrary{arrows}
\usepackage{pgf-umlsd}
\usepackage{microtype}

\usepackage[backend=bibtex,style=ieee,firstinits,maxnames=2,minnames=1]{biblatex}
\addbibresource{biblio.bib}
\setlength{\biblabelsep}{.5em}
\setlength{\bibitemsep}{1pt}

\usepackage{bold-extra}

\newcommand{\Jolie}{Jolie}
\newcommand{\Definition}{\noindent\textbf{\emph{Definition}}}
\newcommand{\Implementation}{\noindent\textbf{\emph{Implementation}}}
\newcommand{\Section}{\S}

\newcommand{\citeNeed}{{\color{red}[CitNeed]}}

\definecolor{color:keyword}{rgb}{0.53,0.05,0.05}
\definecolor{color:comment}{rgb}{0.25,0.37,0.75}
\definecolor{color:string}{rgb}{0.87,0.0,0.0}

\lstdefinelanguage{Jolie}{
	morekeywords={csets,type,raw,any,undefined,void,default,if,for,while,spawn,foreach,else,define,main,include,constants,inputPort,outputPort,interface,execution,cset,nullProcess,RequestResponse,OneWay,throw,throws,install,scope,embedded,init,synchronized,global,is_defined,is_int,is_bool,is_long,is_string,bool,long,int,string,double,undef,with,Location,Protocol,Interfaces,Aggregates,Redirects,linkIn,linkOut},
	sensitive=true,
	morecomment=[l]{//},
	morecomment=[s]{/*}{*/},
	morestring=[b]",
	otherkeywords={;,|,@}
}

\lstset{
	language=Jolie,
	mathescape=true,
	resetmargins=true,
	numberstyle=\footnotesize,
	numbers=left,
	numbersep=5pt,
	numberblanklines=true,
	basicstyle=\ttfamily\small,
	tabsize=2,
	%frame=lines,
	commentstyle=\rmfamily\color{color:comment},
	stringstyle=\color{color:string},
	captionpos=b,
	keywordstyle=\bfseries\color{color:keyword},
	showstringspaces=false,
	belowcaptionskip=10mm,
	breaklines=false,
	columns=fullflexible,
	linewidth= 0.8\linewidth
}

\newcommand{\code}[1]{\lstinline{#1}{}}

\newcommand{\setUmlSeqChartStyle}{
	\tikzset{inststyle/.style={
    rectangle, draw, 
    anchor=west, 
    minimum height=0.8cm, 
    minimum width=1.6cm, 
    fill=white
    %drop shadow={opacity=0,fill=black}]
    }
  }
}

\renewcommand{\mess}[4][0]{
  \stepcounter{seqlevel}
  \path
  (#2)+(0,-\theseqlevel*\unitfactor-0.7*\unitfactor) node (mess from) {};
  \addtocounter{seqlevel}{#1}
  \path
  (#4)+(0,-\theseqlevel*\unitfactor-0.7*\unitfactor) node (mess to) {};
  \draw[->,>=angle 60] (mess from) -- (mess to)% 
    node[midway, above, align=center, text width=3cm]
    {\footnotesize #3};

  \node (#3 from) at (mess from) {};
  \node (#3 to) at (mess to) {};
}

\crefname{figure}{Fig.}{Figs.}
\crefname{lstlisting}{Listing}{Listings}


\usepackage[textsize=tiny,linecolor=teal,%
bordercolor=teal,backgroundcolor=lime!25]{todonotes} % for margin
\setlength{\marginparwidth}{2cm}
  
\title{A Language-based Approach for Interoperability of IoT
Platforms}

\author{Maurizio Gabbrielli, Saverio Giallorenzo, Ivan Lanese, and Stefano Pio
Zingaro 
\\
Universit\`a di Bologna / INRIA \\
{\underline{maurizio.gabbrielli@unibo.it}, 
  \underline{ saverio.giallorenzo@gmail.com}}\\
  {\underline{ivan.lanese@gmail.com},
  \underline{stefanopio.zingaro@unibo.it}
  } \\
}

\date{}

\begin{document}

\maketitle

\begin{abstract}

\normalfont\itshape The Internet of Things (IoT) promotes the communication
among heterogeneous entities, from small sensors to Cloud systems. However,
this is realized using a wide range of communication media and data protocols,
usually incompatible with each other. Thus, IoT systems tend to grow as
homogeneous isolated platforms, which hardly interact. To achieve a higher
degree of interoperability among disparate IoT platforms, we propose a
language-based approach for communication technology integration. We build on
the Jolie programming language, which allows programmers to easily make the
same logic work over disparate communication stacks in a declarative, dynamic
way. Jolie currently supports the main technologies from Service-Oriented
Computing, such as TCP/IP, Bluetooth, and RMI at transport level, and HTTP and
SOAP at application level. As technical result, we integrate in Jolie the two
most adopted protocols for IoT communication, i.e., CoAP and MQTT. In this
paper, we report our experience and we present high-level concepts valuable
both for the general implementation of interoperable systems and for the
development of other language-based solutions.
\end{abstract}

\section{Introduction}
\label{sec:intro}
IoT systems are being developed for a wide range of applications and target
areas~\cite{GubbiBMP13,Atzori20102787}, using a number of different technology stacks~\cite{7123563}.
%
Nevertheless, as reported, e.g., in~\cite{Soursos16,Gojmerac16}, IoT
platforms frequently take the shape of vertical solutions (usually
dubbed ``IoT islands'') that focus on a specific application domain
and rely on a single communication technology stack. Such platforms
provide little support for collaboration and integration. How to
overcome this limitation is currently a hot topic, tackled also by
ongoing EU projects, e.g., symbIoTe~\cite{Gojmerac16} and
bIoTope~\cite{biotope}.

The problem of integration involves many layers, spanning from link-layer
communication technologies, such as ZigBee and WiFi, to application-layer
protocols like HTTP, CoAP~\cite{coap,doi:10.17487/RFC7252}, and
MQTT~\cite{mqtt,mqtt-v3.1.1}, reaching the top-most layers of data-format
integration~\cite{Milenkovic:2015:CII:2843962.2822643}.

In this work, we tackle the main issues of intercommunication among
IoT islands, by focusing on integration at both the transport (TCP or
UDP) and application level. The approach we propose is language based,
that is, we aim at devising a programming language where different
communication protocols can seamlessly coexist and
interoperate. Thanks to proper abstractions provided by the language we
propose, programmers can easily change the transport and application
protocols used for a given communication, even at runtime.
%
Notably, when the application protocol supports different representation formats
(such as JSON, XML, etc.) of the message payload, as in the case of HTTP and
CoAP, the language we propose can automatically marshal and un-marshal data as
required.

Following our approach, most of the complexity of guaranteeing interoperability
among protocols is managed by the programming language, and hidden from the
programmer. This hidden complexity is particularly high when many technologies
are involved. The problem is exacerbated when one has to replace the technology
used for some specific interaction. The replacement may be either static, e.g.,
because of the deployment of new, heterogeneous devices in a pre-existing
system, or dynamic, e.g., to support a changing topology of disparate mobile
devices. These are scenarios where our language-based approach can make the
difference with respect to other approaches.

As an illustrative example, let us consider a scenario where we want to
integrate two islands of IoT devices, both collecting temperature data, but
relying on different communication stacks, namely HTTP over TCP and CoAP over
UDP.
%
We want to program a collector that receives temperature measurements from both
islands and uses them for further elaboration.

In the language we propose, the description of such a collector is divided into
two parts: a \emph{behavior} specifying the logic of the elaboration and a
\emph{deployment} describing in a declarative way how communication is
performed.

The behavior could be of the shape:
%
\begin{lstlisting}[numbers=left,basicstyle=\footnotesize\ttfamily]
main {
  ...
  receiveTemperature( data );
  ...
}
\end{lstlisting}
%
where Line 3 is a reception statement, expecting a temperature measurement on
\emph{operation} \code{receiveTemperature} (an operation is an abstraction
for
technology-specific concepts such as channels, resources, URLs, \dots) and
storing it in variable \code{data}.

Once we defined the logic of the collector, we need to specify on which
technologies the communication happens; in the example above, how the collector
accepts communications from devices. In our language this information is defined
within \emph{ports}. For instance, the port for receiving (denoted with keyword
\code{inputPort}) HTTP measurements can be defined as follows:
%
\begin{lstlisting}[numbers=left,basicstyle=\ttfamily\footnotesize]
interface TemperatureInterface {
 OneWay: receiveTemperature( string )
}

inputPort CollectorPort1 {
 Location: "socket://collector.net:8000"
 Protocol: http
 Interfaces: TemperatureInterface
}
\end{lstlisting}

Port \code{CollectorPort1} specifies that the collector expects inbound
communications via \code{Protocol} \code{http} using a TCP/IP socket receiving
at URL \code{"collector.net"} and on TCP port \code{8000}. The port and the
operation are linked by the definition of \code{interface} 
\code{TemperatureInterface}. The interface declares
the operation \code{receiveTemperature}, including the type of expected
data (\code{string}), as a \code{OneWay} operation, namely an
asynchronous communication that does not require any reply from the collector.

Thanks to port \code{CollectorPort1}, the collector can receive data from the
HTTP island. To integrate the second island, we just need to define an
additional port, similar to \code{CollectorPort1}, except for using UDP/IP
datagrams at the transport layer and CoAP at the application layer. Hence, the
whole code of the collector becomes:
%
\begin{lstlisting}[%
  basicstyle=\footnotesize\ttfamily,
  label=temperature_interface, 
  caption=Code of the Collector Example.]
interface TemperatureInterface {
  OneWay: receiveTemperature( string )
}

inputPort CollectorPort1 {
  Location: "socket://collector.net:8000"
  Protocol: http
  Interfaces: TemperatureInterface
}

inputPort CollectorPort2 {
  Location: "datagram://coap.me:5683"
  Protocol: coap
  Interfaces: TemperatureInterface
}

main {
  receiveTemperature( data );
}
\end{lstlisting}
%
The example above highlights how, using the proposed language abstractions, the
programmer can write a unique behavior and exploit it to receive data sent over
disparate technology stacks. Clearly, one can define different operations for
different technologies, if the required elaborations differ. Our language
supports both inbound and outbound communications, the latter declared with
\code{outputPort}s, whose structure follows that of \code{inputPort}s. In
addition, we let programs change the \code{Location} and \code{Protocol} of its
\code{outputPort}s at runtime, enabling the selection of the appropriate
technologies for each context.

\subsection{Our contribution}
\label{sub:aim_of_research}

To enable the programming of IoT integration in the above style, we do not start
from scratch, but we leverage the work done in the area of Service-Oriented
Architectures (SOAs)~\cite{Erl07} and, in particular, we build on the Jolie
programming language~\cite{MONTESI200719,MGZ07,jolie-lang}. Indeed, the example
above uses the Jolie syntax and abstraction mechanisms.

As mentioned, Jolie enforces a strict separation of concerns between behavior,
describing the logic of the application, and deployment, describing the
communication capabilities. The behavior is defined using the typical constructs
of structured sequential programming, communication primitives, and operators to
deal with concurrency (parallel composition and input choice).

Jolie communication primitives comprise two modalities of
interaction. Outbound \code{OneWay} communications send a message
asynchronously, while \code{RequestResponse} communications send a message
and wait for a reply (they capture the well-known pattern of request-response
interactions~\cite{req-rep}). Dually, inbound \code{OneWay}
communications wait for a message, while \code{RequestResponse}
communications wait for a message and send back a reply.

A main feature of the Jolie language, and the reason why we base our approach on
it, is that it allows one to switch among many communication media (via keyword
\code{Location}) and data protocols (via keyword \code{Protocol}) in a simple,
uniform way.
%
%Indeed, for each communication port the used medium and data protocol
%are declared in the \emph{deployment} part of the program.  In order
%to change the used medium and protocol one just needs to change the
%declaration, without affecting the \emph{behaviour} part, which
%describes the actual computation. Hence, functionalities defined by a
%single behaviour can be made available using different media and
%protocols by specifying different deployments.
Being born in the field of SOAs, Jolie supports the main communication media
(TCP/IP sockets, Bluetooth L2CAP, Java RMI, and Unix local sockets) and data
protocols (HTTP, JSON-RPC, XML-RPC, SOAP and their respective SSL versions) from
this area.

We think that the ability to use different communication modalities in a uniform
way and to easily switch between them is very useful in the area of IoT.
However, to make this approach practical, we also need to support the main
communication stacks used in the IoT setting. Indeed, the main technical
contribution of the present paper is the introduction in Jolie of the support
for two application protocols relevant in the IoT scenario, namely
CoAP~\cite{doi:10.17487/RFC7252,coap} and MQTT~\cite{mqtt-v3.1.1,mqtt}.

Even if Jolie provides support for easy integration of new protocols, the task
is non trivial. Indeed, all the protocols currently integrated in Jolie 
support the same internal interface, based on two assumptions:
\emph{i}) the usage of underlying technologies that ensure reliable
communications and \emph{ii}) a point-to-point communication pattern.

However, both these assumptions fall when considering the two IoT technologies
we integrate:

\begin{itemize}
  \item CoAP communications can be unreliable since they
are based on UDP connectionless datagrams. CoAP provides options for reliable
communications, however these are usually disabled in an IoT setting, since
battery and bandwidth preservation is important.

\item MQTT communications are based on the publish-subscribe
paradigm, which contrasts with Jolie point-to-point communication primitives.
Hence, we need to define a mapping of the general abstractions of the Jolie
language into the publish-subscribe paradigm, balancing two factors: \emph{i})
preserving the simplicity of the point-to-point communication style and
\emph{ii}) capturing the typical flow of communications as programmed in a
publish-subscribe style. An evident example of the challenges of our mapping is
the implementation of request-response communications on top of
publish-subscribe interactions. Remarkably, the mapping that we present in this
work is general and could be used also in other contexts.

\end{itemize}

The flexibility provided by Jolie can be used to support and interconnect
multiple IoT islands, as discussed above. Jolie supports also more advanced
scenarios where the selection of the protocol to use changes according to
internal or environmental conditions, such as available energy or quality of
communication, but of course this requires some capability of switching the
protocol also from the side of the Things, which may not be the case in many
practical situations.

Indeed, in the rest of the paper, we omit to model IoT devices --- like Arduino
and other microcontrollers --- that are at the edge of the network, since they
are normally programmed by using low-level languages.  In principle, these
devices could be programmed by using Jolie-like languages, possibly extending
them to provide those low-level abstractions needed by programmers to access the
in-board sensors and actuators.  However, the constraints on the hardware and
the usually limited amount of energy available to edge devices require a
low-footprint, lightweight execution environment.  Although these requirements
could be achievable also for a language like Jolie, this would require a strong
engineering effort, which is not considered in this paper. We argue that this
direction of work is not urgent, since currently developers tend to program very
simple behaviors for edge devices~\cite{7123563}, which usually capture some
data (e.g., through one of their sensors) and then send them to other devices
(gateways, aggregators, servers). These other devices have more powerful
hardware and less constraints on energy consumption, and can then implement the
logic for the data processing. Hence, here we neglect the programming of edge
devices
%
%would require a considerable change in the actual patterns of interactions that
%are currently by developers. In fact,
%
and we focus on those devices that can both host the Jolie
runtime and whose topological context can benefit from the
flexibility offered by the language.


\section{Jolie for IoT}
\label{sec:challenges}
As mentioned above, Jolie currently supports some of the main technologies used
in SOAs. However, only a limited amount of IoT devices uses the media and
protocols already supported by Jolie. Indeed, protocols such as
CoAP/REST~\cite{doi:10.17487/RFC7252,coap} and MQTT~\cite{mqtt-v3.1.1,mqtt},
which are widely used in IoT scenarios, are not yet implemented in Jolie.
Implementing these protocols is essential in order to allow Jolie programs to
directly interact with the majority of IoT devices. However there are some
challenges linked to the implementation of these technologies within Jolie:

\begin{itemize}
  \item \textit{lossless vs.\@ lossy protocols} --- In SOAs,
  machine-to-machine communication relies on lossless protocols, as there are no
  strict constraints on energy consumption or bandwidth, hence the number of
  message exchanges at the transport level needed for ensuring a message
  delivery is not critical. On the contrary, in IoT networks these
  constraints exist and are important, and the choice of the protocol needs to
  take them into account. Many protocols, and the CoAP application protocol in
  particular, rely on the UDP transport protocol --- a connectionless protocol
  that gives no guarantee on the delivery of messages, but allows one to limit
  message exchanges and energy and bandwidth consumption. Since Jolie assumes
  lossless communications, the inclusion of connectionless protocols in the
  language requires careful handling to prevent misbehaviors;

  \item \textit{point-to-point vs.\@ publish-subscribe} --- In order
  to provide language constructs that do not depend on the chosen protocol, we
  need to find a uniform setting covering both point-to-point communications,
  such as the ones of HTTP and CoAP, and publish-subscribe communications
  typical of MQTT.\@ Jolie already provides language constructs usable with many
  communication protocols, hence the less disruptive approach is to use the same
  constructs, which are designed for a point-to-point setting, also for MQTT.\@
  This requires to find for each point-to-point construct a corresponding effect
  in the publish-subscribe paradigm, such that typical programming patterns
  produce similar effects in both settings. In this way, one can program a
  unique behavior valid for both point-to-point and publish-subscribe
  scenarios.
  
\end{itemize} 

We detail how we integrate CoAP/UDP and MQTT in the Jolie language respectively
in \cref{sub:coap,sub:mqtt}.
%
The Jolie language interpreter, including our extensions at version 1.0, is
available at~\cite{jiot}. The integration of our extension into an official
Jolie release is ongoing work.


\section{Supporting CoAP in Jolie}
\label{sub:coap}
The \textit{Constrained Application Protocol}
(CoAP)~\cite{coap,doi:10.17487/RFC7252} is a specialized web transfer protocol
for constrained scenarios where nodes have low power and networks are lossy. The
goal of CoAP is to import the widely adopted model of REST
architectures~\cite{fielding00} into an IoT setting, that is, optimizing it for
Machine-to-Machine applications. In particular, CoAP makes use of GET, PUT,
POST, and DELETE methods like HTTP.
%
Following the RFC~\cite{doi:10.17487/RFC7252}, CoAP is implemented on top of the
UDP transport protocol~\cite{UDP}, with optional reliability. Indeed, CoAP
provides two communication modalities: a reliable one, obtained by marking the
message type as confirmable (CON), and an unreliable one, obtained by marking the
message type as non confirmable (NON).

As an example, we consider a scenario with a controller, programmed in Jolie,
that communicates with one of many thermostats in a home automation scenario.
Thermostats are accessible at the generic address
\code{"coap://thermostat/##"} where \code{"##"} is a two-digit number
representing the identifier of a specific device. Thermostats accept two
interactions: a GET request on URI \code{"coap://thermostat/##/getTemperature"},
that returns the current temperature, and a POST
request on URI \code{"coap://thermostat/##/setTemperature"}, that sets the
temperature of the HVAC system.
%
We report and comment below the code of a possible Jolie controller.
 % thatinteracts with a specific thermostat.

\begin{lstlisting}[
basicstyle=\footnotesize\ttfamily,
label=lst:coap_example,
caption=Jolie controller communicating over CoAP/UDP.]
type TmpType: void { .id: string } | int { .id: string }

interface ThermostatInterface {
  RequestResponse: getTmp( TmpType )( int )
  OneWay: setTmp( TmpType )
}

outputPort Thermostat {
 Location: "datagram://thermostat:5683"
 Protocol: coap {
  .osc.getTmp << {
    .messageCode = "GET"
    .alias = "/%!{id}/getTemperature",
  };
  .osc.setTmp << {
    .contentFormat = "text/plain",
    .alias = "/%!{id}/setTemperature",
    .messageCode = "POST"
  }
 }
 Interfaces: ThermostatInterface
}

main {
 getTmp@Thermostat( { .id = "42" } )( temp );
 if ( temp > 27 ){
  setTmp@Device( 24 { .id = "42" } )
 } else if ( temp < 15 ){
  setTmp@Device( 22 { .id = "42" } )
 }
}
\end{lstlisting}

Our scenario uses two CoAP resources: \code{"/getTemperature"} and
\code{"/setTemperature"}. 
We model them in Jolie at lines 3--6
of \cref{lst:coap_example}, by defining the \code{interface}
\code{ThermostatInterface}, which includes a
\code{RequestResponse} operation \code{getTmp}, representing resource
\code{"/getTemperature"}, and a \code{OneWay} operation
\code{setTmp}, representing resource \code{"/setTemperature"}.
%
By default, we map operation names to resource names, hence in our example we
would need resources named \code{"/getTmp"} and
\code{"/setTmp"} respectively. However, as described below, one
could override the default mapping, defining the coupling of protocol-specific
concepts (here CoAP resources) and operations inside ports. In this way,
programmers can define interactions at a high level with interfaces, while the
grounding to the specific case is done in the deployment.
% 
% Here we purposefully choose to use operation names that differ from resource
% names to underline that the two concepts are related but loosely coupled.

At lines 8--24 we define an \code{outputPort} to interact with the
\code{Thermostat}.
%
At line 9 we specify the \code{Location} of the thermostat. Recalling that
the scheme of the resources of the thermostats is
\code{"coap://thermostat/##/..."}, we define the \code{Location} of
the port using the UDP \code{"datagram://"} protocol, followed by the
first part of the resource schema \code{"thermostat"} and the UDP port on
which it accepts requests. Here we assume thermostats to use CoAP standard UDP
port, which is \code{"5683"}. Note that, in the \code{Location}, we do
not define the address of a specific thermostat, e.g.,
\code{"datagram://thermostat:5683/42"}. On the contrary, we just specify
the generic address to access thermostats in the system, while the specific
binding will be done at runtime, thanks to the \code{.alias} parameter of
the \code{coap} protocol, described later on.

At line 10 we define \code{coap} to be the protocol used by the 
\code{outputPort}. At lines 11--21 we specify some parameters of the
\code{coap} protocol --- this matches the standard way in which Jolie defines
parameters for \code{Protocol}s in ports.

Here, we follow the methodology presented in~\cite{montesi16} for the
implementation of the HTTP protocol in Jolie --- indeed CoAP adopts HTTP naming
schema and resource interaction methods. In particular, we draw
from~\cite{montesi16} the parameter prefix \code{.osc}, whose name is the
acronym of ``operation-specific configuration'' and which is used for
configuration parameters related to a specific operation.

In the example, we define \code{.osc} parameters for both operations
\code {getTmp} and \code{setTmp}.\@ At line 15 we specify that the CoAP method used is \code{GET}. At
line 14 we define, using the \code{.contentFormat} parameter, that the encoding of
the payload of the message in a binary format (\code{"text/plain"}).
Other accepted values for the
\code{.contentFormat} parameter are \code{"json"} and \code{"xml"}. Marshalling and
un-marshalling is automatic and transparent to the programmer. This feature is
enabled by the structure of Jolie variables, which are always tree-shaped, hence
they can easily be translated into representations based on that shape. At line
15, following the practice introduced in~\cite{montesi16}, we specify that
\code{getTemp} \code{alias}es a resource whose path concatenates a static part,
given by the \code{Location}, and the instantiation of the template
\lstinline|"/%!{id}/getTemperature"| provided by protocol parameter
\code{.alias}. The template is instantiated using values from the parameter of
the operation invocation in the behavior, e.g., value \code{42} at line
27\footnote{In Jolie the dot \code{.} defines path traversals inside trees.
Hence, the notation \code{\{.id = 42\}} indicates a tree with an empty root and
a subnode called \code{id}, whose value is \code{42}.}. Hence, the
interpretation of the declaration at line 15 is that, when invoking operation
\code{getTmp} at runtime, the element \code{id} of the invocation will be
removed from the payload and used to form the address of the requested resource.
The aliasing for operation \code{setTmp} (line 20) is similar to that of
\code{getTmp}, while the operation use
method \code{POST}. Since here the \code{.contentFormat} parameter is omitted, the
default \code{"text/plain"} is used.

To conclude, we briefly comment the runtime execution of the example, described
in the behavior at lines 26--33. At line 27 the controller invokes operation
\code{getTmp}. Being an outgoing \code{RequestResponse}, the
invocation defines on which port to perform the request (\code{Thermostat})
and presents two pairs of round brackets: the first contains the data for the
request, the second points to the variable that will store the received
response. Recalling the aliasing defined at line 15, at line 27 we define the
value of element \code{id = 42}, thus the URI of the resource invoked at
runtime is \code{"coap://thermostat/42/getTemperature"}. Notably, in the
example we hard-coded the \code{id} of the device, however in a realistic
setting the value of \code{id} would be retrieved from a variable. Once
received, the response from thermostat \code{42} is assigned to variable
\code{temp}. The example concludes with a conditional in which, if the
temperature is above \code{27} (line 28), the thermostat is set to lower room
temperature to \code{24} degrees, while, if the temperature lies below \code
{15} degrees, the thermostat is set to raise the temperature to \code{22}
degrees.

Dually to \code{outputPort}s, \code{inputPort}s allow the programmer
to specify inbound communications. The parameters described above are valid also
for \code{inputPort}s, with the only difference that
\code{messageType} works only for \code{RequestResponse}s, and
specifies whether the communication of the reply is reliable or not. 
%
Note that, concerning the \code{.alias} parameter, the template is instantiated
using the address of the incoming communication and the values are inserted
among the elements of the payload.

\subsection{Implementation of CoAP/UDP in Jolie} % (fold)
%
We end this section reporting the most relevant issues met during the
implementation in Jolie of the CoAP/UDP stack.
%
In Jolie the implementations of the supported application and transport
protocols are independent. This enables the composition of any transport
protocol with any application protocol. In particular, the implementation of UDP
that we provide can also be used to support other protocols relying on UDP like
MQTT-SN~\cite{hunkeler08}. For this reason, we separately
present the integration in Jolie of UDP and of CoAP.

Concretely, the Jolie language is written in Java and provides proper abstract
classes that represent application and transport protocols. Each protocol is
obtained as an implementation of the corresponding abstract classes. Each
implementation is a separated module which is loaded only if the protocol is
used. This expedites the integration of new protocols in the language.

%
The implementation of UDP consists in a listener and a
sender class, both based on the Netty framework~\cite{maurer16}.
Since the structure expected by Jolie and the one provided by Netty are similar,
the integration of UDP is smooth. 
% An interesting point is that exceptions raised by Netty are captured
% and transformed into Jolie exceptions. These exceptions are notified to the
% application protocol, which can either manage them or raise them at the level of
% the behavior of the Jolie program.

The implementation of CoAP consists in a unique class, taking care
of both encoding and decoding messages, and is based on nCoAP~\cite{ncoap}.
% an extension of the Netty framework. Also in this case the integration is
% smooth.

We notice that CoAP supports request-response communications and, in particular,
CoAP messages include fields \emph{i}) to specify at which address the reply is
expected and \emph{ii}) to match a reply with a previous request. Hence, the
implementation of \code{RequestResponse} communications in CoAP is sound
also with a transport protocol which is not connection-oriented, such as UDP.\@
This would be a problem for protocols that do not provide such a facility, such
as HTTP, which is indeed not commonly used over UDP.

Notably, Jolie comes with a formal semantics (in terms of a process
calculus)~\cite{Guidi2006}, which enables to rigorously reason on the behavior
of Jolie programs. 
% This has been instrumental in the
% evolution of the language, e.g., to specify and prove properties on the fault
% handling mechanisms of the language~\cite{GuidiLMZ09} or to correctly implement
% sessions\cite{MontesiC11} based on correlation mechanisms\cite{bpel}. 
The semantics in~\cite{Guidi2006} only considers reliable communications and
needs to be extended to also cover the unreliable case. We do not report here on
this topic, since it is not central for the purpose of this paper.


\section{Supporting MQTT in Jolie}
\label{sub:mqtt}
\textit{Message Queue Telemetry Transport} (MQTT)~\cite{mqtt,mqtt-v3.1.1} is a
publish/subscribe messaging application protocol built on top of
the TCP transport protocol. 

A typical publish/subscribe interaction pattern can be diagrammatically
represented as in \cref{fig:pub_sub_encoding_one_way} where:

\begin{enumerate}
  \item a Subscriber subscribes to topic (a) at some Broker;
  \item a Publisher publishes a message to topic (a) at the same Broker;
  \item the Broker forwards the message to topic (a) to the Subscriber.
 \end{enumerate}
 
\begin{figure}[th]
% \hspace{1em}\resizebox{0.4\textwidth}{!}{
\begin{sequencediagram}
\setUmlSeqChartStyle

  \newinst{sub}{Subscriber}
  \newinst[1.8]{bro}{Broker}
  \newinst[1]{pub}{Publisher}

  \mess{sub}{{1) Subscribe to (a)}}{bro}
  \mess{pub}{{2) Publish at (a)}}{bro}
  \mess{bro}{{3) Forward msg in (a)}}{sub}
\end{sequencediagram}
% }
\caption{\label{fig:pub_sub_encoding_one_way}
Typical publish/subscribe interaction pattern.}
\end{figure}

On top of the basic mechanism of publish/subscribe, MQTT defines three levels of
quality of service (QoS) for the delivery of each message published by a
publisher. QoS levels determine whether messages can be lost and/or duplicated.
Concretely, QoS levels are as follows:
\begin{itemize}
  \item \textit{At most once} --- the message can be lost, no duplication
  can occur.
  \item \textit{At least once} --- delivery of the message is guaranteed, but
  duplication may occur.
  \item \textit{Exactly once} --- delivery of the message is guaranteed and
  duplication cannot occur.
\end{itemize}

To present how we model the MQTT protocol in Jolie, we first detail the simpler
case of \code{OneWay} communications in \cref{sub:ow_in_mqtt}. Then, we address
the more complex case of \code{RequestResponse}s in \cref{sub:rr_in_mqtt}. Both
mappings are general, i.e., in principle they can be applied to implement
one-way and request-response communication patterns on top of any
publish/subscribe protocol. Remarks on the implementation are provided in 
\cref{sub:impl_mqtt}.

\subsection{One-Ways in MQTT}
\label{sub:ow_in_mqtt}

We first consider the case of inbound communications and then the case of
outbound communications.

To exemplify \code{OneWay} inbound communications, we use the example in
\cref{lst:mqtt_simple_ex}, which is a revision of the example in
\cref{temperature_interface} by omitting the HTTP and CoAP ports and by adding an
MQTT \code{inputPort}.

\begin{lstlisting}[%
  basicstyle=\footnotesize\ttfamily,
  label=lst:mqtt_simple_ex, 
  caption={Code of the Collector Example, revised for MQTT.}]
interface TemperatureInterface {
  OneWay: receiveTemperature( string )
}

inputPort CollectorPort3 {
 Location: "socket://localhost:8050"
 Protocol: mqtt {
  .broker = "socket://iot.eclipse.org:1883"
 }
 Interfaces: TemperatureInterface
}

main {
  receiveTemperature( data )
}
\end{lstlisting}

As expected, the program behavior and the structure of the \code{inputPort}
are unchanged. Main novelties are:
%
\begin{itemize}
  \item the used \code{Location} (line 6) has the prefix 
  \code{"socket://"} (as seen in the HTTP port) since MQTT relies on TCP
  transport protocol;
  \item the used \code{Protocol} (line 7) is \code{mqtt};
  \item the \code{.broker} protocol parameter (line 8), which is
  compulsory, specifies the address of the Broker.
\end{itemize}

While the syntax and the effect of the communication primitive from the point of
view of the programmer are the same as the ones in \cref{temperature_interface},
the actual message exchanges performed to obtain such an effect are different.

Beyond defining such message exchanges, we also need to decide how to identify
the topic on which the message exchange is performed.

Regarding the message exchanges, from the point of view of the programmer, an
inbound \code{OneWay} communication receives a datum from the communication
partner. To obtain the same effect using the publish/subscribe paradigm, one has
first to subscribe at the Broker to the chosen topic and then wait to receive a
message on that topic, forwarded by the Broker. How topics are selected will be
detailed later on. The execution of a reception on a \code{OneWay} operation
comprises two actual communications: a subscription from the program to the
Broker and a message delivery in the opposite direction. However, subscription
to topics and the execution of a message reception are logically separated and
can be done at different moments. Indeed, the subscription is performed when the
Jolie program is launched for all operations present in MQTT
\code{inputPort}s. This choice is more in line with the expected behavior of
Jolie programs --- and of Service-Oriented programs in general --- where
messages to operations, whose reception statements are not yet enabled, are
stored until the actual execution of the reception.
%
% Here, if the
% subscription is performed along with the execution of the
% \code{OneWay} operation, previous messages could be no more available.
%
In Jolie, the compulsory parameter \code{.broker} is needed precisely to know
the address at which the subscription needs to be performed. The address for the
delivery of the actual message is the usual \code{Location} of the
\code{inputPort}.

Regarding the selection of topics, similarly to what done for CoAP resources, in
MQTT we default to mapping Jolie operations to topics, otherwise we use the
\code{.osc} parameter \code{.alias} to enable loose coupling between operations
and topics. We remark that \code{.alias} in \code{inputPort}s have a different
behavior in MQTT with respect to HTTP and CoAP.\@ In CoAP the name of the resource extracted
from the received message is used to derive the correct instantiation of the
\code{.alias} template. The values resulting from the match are then
inserted among the elements of the payload before storing it in the target
variable \code{data}. Instead, in MQTT, the \code{.alias} parameter is used to
identify the topic for subscription. For example, in \cref{lst:mqtt_simple_ex},
one could add the \code{Protocol} parameter \code{.osc.receiveTemperature.alias
= "temperature"} to specify that the selected topic for operation
\code{receiveTemperature} is \code{"temperature"}. Note that, since
there is no outgoing data, templates in MQTT \code{inputPort}s, such as
\code{"temperature"} in the example, are constants (we require all such
constants defined within the same
\code{inputPort} to be distinct). Having only constant aliases is not a relevant
limitation in the context of IoT, where topics are mostly statically fixed.
Addressing this limitation without disrupting the uniformity of the Jolie
programming model is not trivial and is left as future work.

% \begin{figure}[ht]
% \begin{sequencediagram}
% \setUmlSeqChartStyle

%   \newinst{sub}{Collector}
%   \newinst[2]{bro}{Broker}
%   \newinst[1]{pub}{Device}

%   \mess{sub}{{1) Subscribe to (receiveTemperature)}}{bro}
%   \mess{pub}{{2) Publish on (receiveTemperature)}}{bro}
%   \mess{bro}{{3) Forward msg on (receiveTemperature)}}{sub}
% \end{sequencediagram}
% \caption{\label{fig:mqtt_simple_ex}
% Representation of the example in \cref{lst:mqtt_simple_ex}.}
% \end{figure}

To conclude the mapping of \code{OneWay} operations in MQTT, we consider here
the case of outbound operations, exemplified in \cref{lst:oneway_out}.
%
Outgoing \code{OneWay} operations simply cause the publication of the value
passed as the parameter of the invocation (line 17) at the Broker. The
address of the Broker is defined by the \code{Location} (line 6) of the
\code{outputPort} \code{Broker}. The topic is derived from the name of the
\code{operation} and the parameter of the invocation, using protocol parameter
\code{.alias} as usual. Being an MQTT publication, we specify the
\code{.QoS} protocol parameter (line 10), which selects the QoS level
``Exactly once'' for the operation \code{setTmp}.

\begin{lstlisting}[basicstyle=\footnotesize\ttfamily,caption={Example of
outgoing MQTT \code{OneWay} communication.},label=lst:oneway_out]
interface ThermostatInterface {
 OneWay: setTmp( TmpType )
}

outputPort Broker {
 Location: "socket://iot.eclipse.org:1883"
 Protocol: mqtt {
  .osc.setTmp << {
   .format = "raw",
   .QoS = 2, // exactly once QoS
   .alias = "%!{id}/setTemperature"
 }
}
Interfaces: ThermostatInterface }

main { 
 setTmp@Broker( 24 { .id = "42" } )
}
\end{lstlisting}

\subsection{Request-Responses in MQTT}
\label{sub:rr_in_mqtt}

To discuss \code{RequestResponse} communications, let us consider the
example in \cref{lst:coap_example}, revised in \cref{lst:mqtt_example}
by replacing the CoAP protocol with MQTT. We omit \code{OneWay}
communications and concentrate on the
outbound \code{RequestResponse}. Afterwards, we will also discuss the
dual inbound \code{RequestResponse}.

\begin{lstlisting}[
basicstyle=\footnotesize\ttfamily,
label=lst:mqtt_example,
caption=Jolie controller communicating over MQTT.]
interface ThermostatInterface {
  RequestResponse: getTmp( TmpType )( int )
}

outputPort Broker {
 Location: "socket://iot.eclipse.org:1883"
 Protocol: mqtt {
  .osc.getTmp << {
   .format = "raw",
   .QoS = 2, // exactly once QoS
   .alias = "%!{id}/getTemperature",
   .aliasResponse = "%!{id}/getTempReply",
  }
 }
 Interfaces: ThermostatInterface
}

main {
 getTmp@Broker( { .id = "42" } )( temp )
}
\end{lstlisting}

Syntactically, the main novelty with respect to the \code{outputPort} in
\cref{lst:oneway_out} is the addition of \code{Protocol} parameter
\code{.aliasResponse}. This parameter specifies the name of the topic
where the receiver will publish its response.

From the point of view of the programmer, an outbound \code{RequestResponse} is
composed of an outgoing communication followed by an inbound reply. The outgoing
communication is implemented using the approach already seen for \code{OneWay}
communications, i.e., using the \code{.alias} \code{Protocol} parameter to
identify the topic. Then, one has the issue of relating the outgoing request
with its reply. Many standard point-to-point communication technologies, such as
HTTP/TCP and the already discussed CoAP/UDP, support request-response
communications by defining means to link a given outgoing request to its reply.
MQTT does not provide dedicated means to do such a linking, hence we exploit the
content and the topic of messages to this end.
%
We identify the topic for the reply with the
\code{.aliasResponse} \code{Protocol} parameter. Like for \code{.alias}
parameters, the template of the \code{.aliasResponse} parameter is instantiated
using the content of the message sent in the behavior. For example, in
\cref{lst:mqtt_example}, we use \code{.id} in line 19 to obtain
\code{"42/getTemperature"} and \code{"42/getTempReply"}, respectively the
publication and reply topics.

We can now describe the pattern of interactions that we use to implement the
outgoing \code{RequestResponse} communication at line 19 in
\cref{lst:mqtt_example}. As a reference, the pattern of interactions is depicted
in the left part of \cref{fig:pub_sub_RR}. We will describe the right part later
on, after having introduced inbound request-response communications.

\begin{figure}[ht]
% \hspace{1em}\resizebox{0.4\textwidth}{!}{
\begin{sequencediagram}
\setUmlSeqChartStyle

  \newinst{pro}{Controller}
  \newinst[1.4]{bro}{Broker}
  \newinst[1.4]{the}{Thermostat}

  \postlevel
  \mess{the}{{1) Subscribe to "42/getTemperature"}}{bro}
  \stepcounter{seqlevel}
  \mess{pro}{{2) Subscribe to "42/getTempReply"}}{bro}
  \postlevel
  \mess{pro}{{3) Publish to "42/getTemperature"}}{bro}
  \mess{bro}{{4) Forward msg in "42/getTemperature"}}{the}
  \postlevel
  \mess{the}{{5) Publish to "42/getTempReply"}}{bro}
  \mess{bro}{{6) Forward msg in "42/getTempReply"}}{pro}

\end{sequencediagram}
% }
\caption{\label{fig:pub_sub_RR} Interaction in the home automation
example in MQTT.}
\end{figure}

First, the controller subscribes to the reply topic 
\code{"42/getTempReply"} at the Broker. Then, the controller
sends to the Broker the request message on topic
\code{"42/getTemperature"}; the payload of the message contains the reply topic
\code{"42/getTempReply"}. The execution of the \code{RequestResponse}
terminates when the Broker forwards the reply received on topic
\code{"42/getTempReply"} to the controller.

Differently from inbound \code{OneWay} communications, here we do not subscribe
to the reply topic when the program is launched. Indeed, it would be useless
since no relevant message can arrive on this topic before the controller sends
its message to the Broker and, by anticipating the subscription, it would
complicate the usage of runtime information in templates.

To exemplify inbound \code{RequestResponse} communications, we assume that the
thermostat in our example is programmed in Jolie. We report its code in
\cref{lst:in_mqtt_example}.

\begin{lstlisting}[
basicstyle=\footnotesize\ttfamily,
label=lst:in_mqtt_example,
caption=Jolie thermostat communicating over MQTT.]
interface ThermostatInterface {
  RequestResponse: getTmp( TmpType )( TmpType )
}

inputPort Thermostat {
 Location: "socket://localhost:9000"
 Protocol: mqtt {
  .broker = "socket://iot.eclipse.org:1883";
  .osc.getTmp << {
   .format = "raw",
   .alias = "42/getTemperature"
  }
 }
 Interfaces: ThermostatInterface 
}

main {
 getTmp( temp )( temp ){
  // retrieves temperature and stores
  // it within the root of variable temp
 }
}
\end{lstlisting}

At line 11 in \cref{lst:in_mqtt_example}, the \code{.alias} parameter
\code{"42/getTemperature"} is static, as usual for \code{inputPort}s.

When the thermostat program is launched, it subscribes to topic
\code{"42/getTemperature"}. When a message on this topic arrives, the reply
topic, e.g., \code{"42/getTempReply"}, is extracted and the rest of the payload
(empty in this case) is passed to the behavior. The body of the
\code{RequestResponse} (lines 19--20) is executed to compute the return value.
Finally, the return value is published on the reply topic
\code{"42/getTempReply"}.

We now summarize the exchange between the controller and the thermostat (left
part of \cref{fig:pub_sub_RR}):

\begin{enumerate}
  \item when the thermostat is started, it subscribes to topic \code{"42/getTemperature"} at the
  Broker;

  \item when the outgoing \code{RequestResponse} is executed, the controller
  subscribes to topic \code{"42/getTempReply"} at the Broker;

  \item the controller publishes the request message to topic
  \code{"42/getTemperature"};

  \item the Broker forwards the message in topic \code{"42/getTemperature"} to
  the thermostat;

  \item the thermostat computes the response and publishes it at topic 
  \code{"42/getTempReply"};

  \item the Broker forwards the message in topic \code{"42/getTempReply"} to the
  controller.

\end{enumerate}

We remark that \code{RequestResponse} operations are meant to be one-to-one
communications. To ensure this in a publish/subscribe setting while using the
approach above, one has to ensure that no other participant subscribes to the
selected topics, which essentially act as namespaces.

\subsection{Implementation of MQTT in Jolie.} % (fold)
\label{sub:impl_mqtt}

The implementation of MQTT in Jolie required the creation of two main classes:
the actual MQTT protocol and a generic publish/subscribe meta-channel that
bridges between the end-to-end style of Jolie communications and
publish/subscribe interactions. Furthermore, we had to update the Jolie class
for channel creation so that it could choose between the standard end-to-end
media and the new publish/subscribe meta-channel.

The MQTT protocol class both encodes and decodes messages and implements the QoS
policies of the MQTT standard. Concretely, as for CoAP, we based the
implementation of MQTT on Netty~\cite{maurer16}. The main difficulty in the
implementation of the protocol is the definition of the message patterns needed
to implement \code{OneWay} and \code{RequestResponse} communications, which have
been described above. Beyond being invoked when operations are executed, the
MQTT class is also invoked when the program is started, to perform port
initialization. In particular, this is when subscriptions to topics identified
in \code{inputPort}s are performed (along with the related connections to the
Brokers).

In addition to the MQTT protocol, we also implemented a generic
publish/subscribe meta-channel. Indeed, since Jolie is based on an end-to-end
communication pattern, it assumes that the caller requires the creation of a
connection to the server, which waits for inbound requests. For this reason,
given a certain medium, \code{inputPort}s and \code{outputPort}s
use a medium-specific implementation of, respectively,
a listener class and a sender class.

%
This pattern, separating listeners from senders, does not apply to
publish/subscribe protocols, where both the subscriber and the publisher need to
establish a connection with the broker. For this reason we use the
publish/subscribe meta-channel to bridge between the two styles. On the one
hand, it implements the interfaces of listeners and senders as required by
Jolie. On the other hand, it relies on the pre-existing Jolie sender classes
(TCP socket in the case of MQTT) to create the connection to the broker.

\section{Related Work}
\label{sec:related}
In the literature there are many proposals for platforms, middlewares, smart
gateways, and general systems, all aimed at solving the interoperability
problem arising from the current ``babel'' of IoT technologies (protocols,
formats, and languages). Without any claim of being complete, here we mention a
few notable examples which are somehow related to our current research.

Recently the W3C started the Web of Things (WoT) Working Group~\cite{w3c17}. The
aim of WoT is to define a standard stack of layered technologies, as well as
software architectural styles and programming patterns, to uniform and simplify
the creation of IoT applications. In this context, the W3C is working on a WoT
Architecture~\cite{wot.arch}. The main concept of the architecture is the notion
of ``servient'', a virtual entity that represents a physical IoT device.
Servients provide technology-independent, standard APIs that developers can use
to transparently operate in heterogeneous environments. Remarkably, both the
WoT proposal and ours concern high-level
abstractions for low-level access to devices provided via, e.g., HTTP, CoAP, and
MQTT.\@ However, while we propose a dedicated language, they provide API
specifications.
%
More in general, there are many proposals for the integration of WoT and IoT.\@
For example~\cite{dominique2011web} and~\cite{corredor2014lightweight} define
general platforms covering different layers of IoT, including an accessibility
layer which integrates concepts like smart gateways and proxies to facilitate
the connection of (smart) Things into the Internet infrastructure, using
architectural principles based on REST.
%
Smart gateways and proxies are used in several industrial proposals to
facilitate the development of applications. Common denominator of some of these
proposals, e.g.,~\cite{60, 61, 62}, is the abstraction of low-level
functionalities provided by embedded devices (e.g., connectivity and
communication over low-level protocols like ZigBee, Z-Wave, Wi/IP/UPnP, etc.).
Smart gateways are used also to translate (or integrate) CoAP into
HTTP~\cite{7811451,s150101217,7037719} and to integrate both CoAP and MQTT by
means of specific middlewares~\cite{6827678}. Eclipse IoT~\cite{EplicseIoT} is
an IoT integration framework proposed by the Eclipse IoT Working Group. Aim of
Eclipse IoT is to build an open IoT stack for Java, including the support for
device-to-device and device-to-server protocols, as well as the provision of
protocols, frameworks, and services for device management. There exist several
 European projects, notably INTER-IoT~\cite{7471373} and
symbIoTe~\cite{Gojmerac16}, that address the issue of interoperability in IoT
and have produced several concrete proposals. Finally, a work close to ours
is~\cite{Zhiliang11}, where a middleware converts IoT heterogeneous networks
into a single homogeneous network.

Although related to our aim in this paper, the cited proposals tackle the
problem of IoT integration from a framework perspective: they provide chains of
tools, each addressing a specific level of the integration stack. Differently,
we extend a language specifically tailored for system integration and advanced
flow manipulation, Jolie, to support integration of IoT devices. This offers a
single linguistic domain to seamlessly integrate disparate low-level IoT devices
and intermediate nodes (collectors, aggregators, gateways). Moreover, Jolie is
already successfully used for building Cloud-based, microservice
solutions~\cite{GabbrielliGGMM16,MelisPGC17}. This makes the language useful
also for assembling advanced architectures for IoT, e.g., to handle real-time
streaming and processing of data from many devices. The benefit, here, is that,
while solutions based on frameworks require dedicated proficiencies on each of
the included tools, Jolie programmers can directly work at any level of the IoT
stack, without the need to acquire specific knowledge on the tools in a given
framework.

To conclude our revision of related work, we narrow our focus on language-based
integration solutions for IoT. The work most related to ours is
SensorML~\cite{SensorML}.
%
SensorML, abbreviation of Sensor Model Language, is a modeling
language for the description of sensors and, more in general, of measurement
processes. Some features modeled by the language are: discovery and
geolocalization of sensors, processing of sensor observations, and
functionalities to program sensors and to subscribe to sensor events.
%
While some traits of SensorML are common to our proposal, the scopes of the two
languages sensibly differ. Indeed, while Jolie is a high-level language for
programming generic architectures (spanning from cloud-based microservices to
low-level IoT integrators), SensorML just models IoT devices, their discovery,
and the processing of sensor observations.

%


\vspace{1em}
\section{Discussion and Conclusion}
\label{conclusion}
In this paper, we proposed a language-based approach for the integration of
disparate IoT platforms. We built our treatment on the Jolie programming
language. This first result is an initial step towards a more comprehensive
solution for IoT ecosystem integration and management. Concretely, we included
in Jolie the support for two of the most widely used IoT protocols. The
inclusion enables Jolie programmers to interact with the majority of present IoT
devices. Summarizing our results: \emph{i}) we included in Jolie the CoAP
application protocol, also extending the Jolie language to support the UDP
transport protocol, \emph{ii}) we added the support for the MQTT protocol and,
in doing so, \emph{iii}) we tackled the challenging problem of mapping the
renowned pattern of request-responses (typical of HTTP and other widely used
protocols) into the publish/subscribe message pattern of MQTT.\@ The mapping
abstracts from peculiarities of MQTT and is applicable to any publish/subscribe
protocol.

Regarding future work, we are currently investigating the integration in Jolie
of more IoT protocols~\cite{7123563}, in order to extend the usability of the
language in the IoT setting.

Another interesting direction comes from the inclusion of publish/subscribe
protocols in Jolie. Indeed, the publish/subscribe pattern is renowned for
enabling high scalability of networks, as well as supporting flexible and highly
dynamic network topologies~\cite{eugster03}. Our intuition is that, besides
efficiency and scalability, publish/subscribe architectures can achieve a higher
degree of reliability if programmed using the Jolie language. 
% Moreover, the
% communication abstractions provided by the language avoid the use of callbacks,
% which are difficult to program, debug, and are one of the main causes of errors
% in concurrent and distributed systems~\cite{fukuda15}.

Another interesting direction for future developments is studying how Jolie
can support the testing of IoT technologies, e.g., to test how
different protocol stacks perform over a given IoT topology. Thanks to the
simplicity of changing the combination of the used protocols (application and
transport), experimenters can quickly test many configurations, also enjoying
a more reliable platform to compare them. Indeed, usually even changing one
of the protocols in the configured stack would require an almost complete
rewrite of the logic of network components. Contrarily, in Jolie, this change
just requires an update of the deployment part of programs, leaving the logic
unaffected. Moreover, such an update could even be done programmatically,
making the practice of repeated experimenting on IoT networks easier and more
standardized.

Finally, as future work, we also consider the possibility of developing a
light-weight version of the language, to be used on low-power IoT devices.
Indeed, in this paper, we assumed that these devices are programmed with
low-level languages, since they can support only a very constrained execution
environment. Clearly, letting programmers develop all the components of an IoT
network in the same language would not only ease its implementation but also
testability, deployment, and maintenance. However, achieving such a result
would require a very challenging engineering endeavor.

% \printbibliography
% \bibliographystyle{ieeetr}
% {\raggedright\bibliography{biblio}}

\printbibliography

\end{document}
