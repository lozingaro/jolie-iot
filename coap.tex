The \textit{Constrained Application Protocol}
(CoAP)~\cite{coap,doi:10.17487/RFC7252} is a specialized web transfer protocol
for constrained scenarios where nodes have low power and networks are lossy. The
goal of CoAP is to import the widely adopted model of REST
architectures~\cite{fielding00} into an IoT setting, that is, optimizing it for
Machine-to-Machine applications. In particular, CoAP makes use of GET, PUT,
POST, and DELETE methods like HTTP.
%
Following the RFC~\cite{doi:10.17487/RFC7252}, CoAP is implemented on top of the
UDP transport protocol~\cite{UDP}, with optional reliability. Indeed, CoAP
provides two communication modalities: a reliable one, obtained by marking the
message type as confirmable (CON), and an unreliable one, obtained by marking the
message type as non confirmable (NON).

As an example, we consider a scenario with a controller, programmed in Jolie,
that communicates with one of many thermostats in a home automation scenario.
Thermostats are accessible at the generic address
\code{"coap://thermostat/##"} where \code{"##"} is a two-digit number
representing the identifier of a specific device. Thermostats accept two
interactions: a GET request on URI \code{"coap://thermostat/##/getTemperature"},
that returns the current temperature, and a POST
request on URI \code{"coap://thermostat/##/setTemperature"}, that sets the
temperature of the HVAC system.
%
We report and comment below the code of a possible Jolie controller.
 % thatinteracts with a specific thermostat.

\begin{lstlisting}[
basicstyle=\footnotesize\ttfamily,
label=lst:coap_example,
caption=Jolie controller communicating over CoAP/UDP.]
type TmpType: void { .id: string } | int { .id: string }

interface ThermostatInterface {
  RequestResponse: getTmp( TmpType )( int )
  OneWay: setTmp( TmpType )
}

outputPort Thermostat {
 Location: "datagram://thermostat:5683"
 Protocol: coap {
  .osc.getTmp << {
    .messageCode = "GET"
    .alias = "/%!{id}/getTemperature",
  };
  .osc.setTmp << {
    .contentFormat = "text/plain",
    .alias = "/%!{id}/setTemperature",
    .messageCode = "POST"
  }
 }
 Interfaces: ThermostatInterface
}

main {
 getTmp@Thermostat( { .id = "42" } )( temp );
 if ( temp > 27 ){
  setTmp@Device( 24 { .id = "42" } )
 } else if ( temp < 15 ){
  setTmp@Device( 22 { .id = "42" } )
 }
}
\end{lstlisting}

Our scenario uses two CoAP resources: \code{"/getTemperature"} and
\code{"/setTemperature"}. 
We model them in Jolie at lines 3--6
of \cref{lst:coap_example}, by defining the \code{interface}
\code{ThermostatInterface}, which includes a
\code{RequestResponse} operation \code{getTmp}, representing resource
\code{"/getTemperature"}, and a \code{OneWay} operation
\code{setTmp}, representing resource \code{"/setTemperature"}.
%
By default, we map operation names to resource names, hence in our example we
would need resources named \code{"/getTmp"} and
\code{"/setTmp"} respectively. However, as described below, one
could override the default mapping, defining the coupling of protocol-specific
concepts (here CoAP resources) and operations inside ports. In this way,
programmers can define interactions at a high level with interfaces, while the
grounding to the specific case is done in the deployment.
% 
% Here we purposefully choose to use operation names that differ from resource
% names to underline that the two concepts are related but loosely coupled.

At lines 8--24 we define an \code{outputPort} to interact with the
\code{Thermostat}.
%
At line 9 we specify the \code{Location} of the thermostat. Recalling that
the scheme of the resources of the thermostats is
\code{"coap://thermostat/##/..."}, we define the \code{Location} of
the port using the UDP \code{"datagram://"} protocol, followed by the
first part of the resource schema \code{"thermostat"} and the UDP port on
which it accepts requests. Here we assume thermostats to use CoAP standard UDP
port, which is \code{"5683"}. Note that, in the \code{Location}, we do
not define the address of a specific thermostat, e.g.,
\code{"datagram://thermostat:5683/42"}. On the contrary, we just specify
the generic address to access thermostats in the system, while the specific
binding will be done at runtime, thanks to the \code{.alias} parameter of
the \code{coap} protocol, described later on.

At line 10 we define \code{coap} to be the protocol used by the 
\code{outputPort}. At lines 11--21 we specify some parameters of the
\code{coap} protocol --- this matches the standard way in which Jolie defines
parameters for \code{Protocol}s in ports.

Here, we follow the methodology presented in~\cite{montesi16} for the
implementation of the HTTP protocol in Jolie --- indeed CoAP adopts HTTP naming
schema and resource interaction methods. In particular, we draw
from~\cite{montesi16} the parameter prefix \code{.osc}, whose name is the
acronym of ``operation-specific configuration'' and which is used for
configuration parameters related to a specific operation.

In the example, we define \code{.osc} parameters for both operations
\code {getTmp} and \code{setTmp}.\@ At line 15 we specify that the CoAP method used is \code{GET}. At
line 14 we define, using the \code{.contentFormat} parameter, that the encoding of
the payload of the message in a binary format (\code{"text/plain"}).
Other accepted values for the
\code{.contentFormat} parameter are \code{"json"} and \code{"xml"}. Marshalling and
un-marshalling is automatic and transparent to the programmer. This feature is
enabled by the structure of Jolie variables, which are always tree-shaped, hence
they can easily be translated into representations based on that shape. At line
15, following the practice introduced in~\cite{montesi16}, we specify that
\code{getTemp} \code{alias}es a resource whose path concatenates a static part,
given by the \code{Location}, and the instantiation of the template
\lstinline|"/%!{id}/getTemperature"| provided by protocol parameter
\code{.alias}. The template is instantiated using values from the parameter of
the operation invocation in the behavior, e.g., value \code{42} at line
27\footnote{In Jolie the dot \code{.} defines path traversals inside trees.
Hence, the notation \code{\{.id = 42\}} indicates a tree with an empty root and
a subnode called \code{id}, whose value is \code{42}.}. Hence, the
interpretation of the declaration at line 15 is that, when invoking operation
\code{getTmp} at runtime, the element \code{id} of the invocation will be
removed from the payload and used to form the address of the requested resource.
The aliasing for operation \code{setTmp} (line 20) is similar to that of
\code{getTmp}, while the operation use
method \code{POST}. Since here the \code{.contentFormat} parameter is omitted, the
default \code{"text/plain"} is used.

To conclude, we briefly comment the runtime execution of the example, described
in the behavior at lines 26--33. At line 27 the controller invokes operation
\code{getTmp}. Being an outgoing \code{RequestResponse}, the
invocation defines on which port to perform the request (\code{Thermostat})
and presents two pairs of round brackets: the first contains the data for the
request, the second points to the variable that will store the received
response. Recalling the aliasing defined at line 15, at line 27 we define the
value of element \code{id = 42}, thus the URI of the resource invoked at
runtime is \code{"coap://thermostat/42/getTemperature"}. Notably, in the
example we hard-coded the \code{id} of the device, however in a realistic
setting the value of \code{id} would be retrieved from a variable. Once
received, the response from thermostat \code{42} is assigned to variable
\code{temp}. The example concludes with a conditional in which, if the
temperature is above \code{27} (line 28), the thermostat is set to lower room
temperature to \code{24} degrees, while, if the temperature lies below \code
{15} degrees, the thermostat is set to raise the temperature to \code{22}
degrees.

Dually to \code{outputPort}s, \code{inputPort}s allow the programmer
to specify inbound communications. The parameters described above are valid also
for \code{inputPort}s, with the only difference that
\code{messageType} works only for \code{RequestResponse}s, and
specifies whether the communication of the reply is reliable or not. 
%
Note that, concerning the \code{.alias} parameter, the template is instantiated
using the address of the incoming communication and the values are inserted
among the elements of the payload.

\subsection{Implementation of CoAP/UDP in Jolie} % (fold)
%
We end this section reporting the most relevant issues met during the
implementation in Jolie of the CoAP/UDP stack.
%
In Jolie the implementations of the supported application and transport
protocols are independent. This enables the composition of any transport
protocol with any application protocol. In particular, the implementation of UDP
that we provide can also be used to support other protocols relying on UDP like
MQTT-SN~\cite{hunkeler08}. For this reason, we separately
present the integration in Jolie of UDP and of CoAP.

Concretely, the Jolie language is written in Java and provides proper abstract
classes that represent application and transport protocols. Each protocol is
obtained as an implementation of the corresponding abstract classes. Each
implementation is a separated module which is loaded only if the protocol is
used. This expedites the integration of new protocols in the language.

%
The implementation of UDP consists in a listener and a
sender class, both based on the Netty framework~\cite{maurer16}.
Since the structure expected by Jolie and the one provided by Netty are similar,
the integration of UDP is smooth. 
% An interesting point is that exceptions raised by Netty are captured
% and transformed into Jolie exceptions. These exceptions are notified to the
% application protocol, which can either manage them or raise them at the level of
% the behavior of the Jolie program.

The implementation of CoAP consists in a unique class, taking care
of both encoding and decoding messages, and is based on nCoAP~\cite{ncoap}.
% an extension of the Netty framework. Also in this case the integration is
% smooth.

We notice that CoAP supports request-response communications and, in particular,
CoAP messages include fields \emph{i}) to specify at which address the reply is
expected and \emph{ii}) to match a reply with a previous request. Hence, the
implementation of \code{RequestResponse} communications in CoAP is sound
also with a transport protocol which is not connection-oriented, such as UDP.\@
This would be a problem for protocols that do not provide such a facility, such
as HTTP, which is indeed not commonly used over UDP.

Notably, Jolie comes with a formal semantics (in terms of a process
calculus)~\cite{Guidi2006}, which enables to rigorously reason on the behavior
of Jolie programs. 
% This has been instrumental in the
% evolution of the language, e.g., to specify and prove properties on the fault
% handling mechanisms of the language~\cite{GuidiLMZ09} or to correctly implement
% sessions\cite{MontesiC11} based on correlation mechanisms\cite{bpel}. 
The semantics in~\cite{Guidi2006} only considers reliable communications and
needs to be extended to also cover the unreliable case. We do not report here on
this topic, since it is not central for the purpose of this paper.
