IoT systems are being developed for a wide range of applications and target
areas~\cite{GubbiBMP13,Atzori20102787}, using a number of different technology stacks~\cite{7123563}.
%
Nevertheless, as reported, e.g., in~\cite{Soursos16,Gojmerac16}, IoT
platforms frequently take the shape of vertical solutions (usually
dubbed ``IoT islands'') that focus on a specific application domain
and rely on a single communication technology stack. Such platforms
provide little support for collaboration and integration. How to
overcome this limitation is currently a hot topic, tackled also by
ongoing EU projects, e.g., symbIoTe~\cite{Gojmerac16} and
bIoTope~\cite{biotope}.

The problem of integration involves many layers, spanning from link-layer
communication technologies, such as ZigBee and WiFi, to application-layer
protocols like HTTP, CoAP~\cite{coap,doi:10.17487/RFC7252}, and
MQTT~\cite{mqtt,mqtt-v3.1.1}, reaching the top-most layers of data-format
integration~\cite{Milenkovic:2015:CII:2843962.2822643}.

In this work, we tackle the main issues of intercommunication among
IoT islands, by focusing on integration at both the transport (TCP or
UDP) and application level. The approach we propose is language based,
that is, we aim at devising a programming language where different
communication protocols can seamlessly coexist and
interoperate. Thanks to proper abstractions provided by the language we
propose, programmers can easily change the transport and application
protocols used for a given communication, even at runtime.
%
Notably, when the application protocol supports different representation formats
(such as JSON, XML, etc.) of the message payload, as in the case of HTTP and
CoAP, the language we propose can automatically marshal and un-marshal data as
required.

Following our approach, most of the complexity of guaranteeing interoperability
among protocols is managed by the programming language, and hidden from the
programmer. This hidden complexity is particularly high when many technologies
are involved. The problem is exacerbated when one has to replace the technology
used for some specific interaction. The replacement may be either static, e.g.,
because of the deployment of new, heterogeneous devices in a pre-existing
system, or dynamic, e.g., to support a changing topology of disparate mobile
devices. These are scenarios where our language-based approach can make the
difference with respect to other approaches.

As an illustrative example, let us consider a scenario where we want to
integrate two islands of IoT devices, both collecting temperature data, but
relying on different communication stacks, namely HTTP over TCP and CoAP over
UDP.
%
We want to program a collector that receives temperature measurements from both
islands and uses them for further elaboration.

In the language we propose, the description of such a collector is divided into
two parts: a \emph{behavior} specifying the logic of the elaboration and a
\emph{deployment} describing in a declarative way how communication is
performed.

The behavior could be of the shape:
%
\begin{lstlisting}[numbers=left,basicstyle=\footnotesize\ttfamily]
main {
  ...
  receiveTemperature( data );
  ...
}
\end{lstlisting}
%
where Line 3 is a reception statement, expecting a temperature measurement on
\emph{operation} \code{receiveTemperature} (an operation is an abstraction
for
technology-specific concepts such as channels, resources, URLs, \dots) and
storing it in variable \code{data}.

Once we defined the logic of the collector, we need to specify on which
technologies the communication happens; in the example above, how the collector
accepts communications from devices. In our language this information is defined
within \emph{ports}. For instance, the port for receiving (denoted with keyword
\code{inputPort}) HTTP measurements can be defined as follows:
%
\begin{lstlisting}[numbers=left,basicstyle=\ttfamily\footnotesize]
interface TemperatureInterface {
 OneWay: receiveTemperature( string )
}

inputPort CollectorPort1 {
 Location: "socket://collector.net:8000"
 Protocol: http
 Interfaces: TemperatureInterface
}
\end{lstlisting}

Port \code{CollectorPort1} specifies that the collector expects inbound
communications via \code{Protocol} \code{http} using a TCP/IP socket receiving
at URL \code{"collector.net"} and on TCP port \code{8000}. The port and the
operation are linked by the definition of \code{interface} 
\code{TemperatureInterface}. The interface declares
the operation \code{receiveTemperature}, including the type of expected
data (\code{string}), as a \code{OneWay} operation, namely an
asynchronous communication that does not require any reply from the collector.

Thanks to port \code{CollectorPort1}, the collector can receive data from the
HTTP island. To integrate the second island, we just need to define an
additional port, similar to \code{CollectorPort1}, except for using UDP/IP
datagrams at the transport layer and CoAP at the application layer. Hence, the
whole code of the collector becomes:
%
\begin{lstlisting}[%
  basicstyle=\footnotesize\ttfamily,
  label=temperature_interface, 
  caption=Code of the Collector Example.]
interface TemperatureInterface {
  OneWay: receiveTemperature( string )
}

inputPort CollectorPort1 {
  Location: "socket://collector.net:8000"
  Protocol: http
  Interfaces: TemperatureInterface
}

inputPort CollectorPort2 {
  Location: "datagram://coap.me:5683"
  Protocol: coap
  Interfaces: TemperatureInterface
}

main {
  receiveTemperature( data );
}
\end{lstlisting}
%
The example above highlights how, using the proposed language abstractions, the
programmer can write a unique behavior and exploit it to receive data sent over
disparate technology stacks. Clearly, one can define different operations for
different technologies, if the required elaborations differ. Our language
supports both inbound and outbound communications, the latter declared with
\code{outputPort}s, whose structure follows that of \code{inputPort}s. In
addition, we let programs change the \code{Location} and \code{Protocol} of its
\code{outputPort}s at runtime, enabling the selection of the appropriate
technologies for each context.

\subsection{Our contribution}
\label{sub:aim_of_research}

To enable the programming of IoT integration in the above style, we do not start
from scratch, but we leverage the work done in the area of Service-Oriented
Architectures (SOAs)~\cite{Erl07} and, in particular, we build on the Jolie
programming language~\cite{MONTESI200719,MGZ07,jolie-lang}. Indeed, the example
above uses the Jolie syntax and abstraction mechanisms.

As mentioned, Jolie enforces a strict separation of concerns between behavior,
describing the logic of the application, and deployment, describing the
communication capabilities. The behavior is defined using the typical constructs
of structured sequential programming, communication primitives, and operators to
deal with concurrency (parallel composition and input choice).

Jolie communication primitives comprise two modalities of
interaction. Outbound \code{OneWay} communications send a message
asynchronously, while \code{RequestResponse} communications send a message
and wait for a reply (they capture the well-known pattern of request-response
interactions~\cite{req-rep}). Dually, inbound \code{OneWay}
communications wait for a message, while \code{RequestResponse}
communications wait for a message and send back a reply.

A main feature of the Jolie language, and the reason why we base our approach on
it, is that it allows one to switch among many communication media (via keyword
\code{Location}) and data protocols (via keyword \code{Protocol}) in a simple,
uniform way.
%
%Indeed, for each communication port the used medium and data protocol
%are declared in the \emph{deployment} part of the program.  In order
%to change the used medium and protocol one just needs to change the
%declaration, without affecting the \emph{behaviour} part, which
%describes the actual computation. Hence, functionalities defined by a
%single behaviour can be made available using different media and
%protocols by specifying different deployments.
Being born in the field of SOAs, Jolie supports the main communication media
(TCP/IP sockets, Bluetooth L2CAP, Java RMI, and Unix local sockets) and data
protocols (HTTP, JSON-RPC, XML-RPC, SOAP and their respective SSL versions) from
this area.

We think that the ability to use different communication modalities in a uniform
way and to easily switch between them is very useful in the area of IoT.
However, to make this approach practical, we also need to support the main
communication stacks used in the IoT setting. Indeed, the main technical
contribution of the present paper is the introduction in Jolie of the support
for two application protocols relevant in the IoT scenario, namely
CoAP~\cite{doi:10.17487/RFC7252,coap} and MQTT~\cite{mqtt-v3.1.1,mqtt}.

Even if Jolie provides support for easy integration of new protocols, the task
is non trivial. Indeed, all the protocols currently integrated in Jolie 
support the same internal interface, based on two assumptions:
\emph{i}) the usage of underlying technologies that ensure reliable
communications and \emph{ii}) a point-to-point communication pattern.

However, both these assumptions fall when considering the two IoT technologies
we integrate:

\begin{itemize}
  \item CoAP communications can be unreliable since they
are based on UDP connectionless datagrams. CoAP provides options for reliable
communications, however these are usually disabled in an IoT setting, since
battery and bandwidth preservation is important.

\item MQTT communications are based on the publish-subscribe
paradigm, which contrasts with Jolie point-to-point communication primitives.
Hence, we need to define a mapping of the general abstractions of the Jolie
language into the publish-subscribe paradigm, balancing two factors: \emph{i})
preserving the simplicity of the point-to-point communication style and
\emph{ii}) capturing the typical flow of communications as programmed in a
publish-subscribe style. An evident example of the challenges of our mapping is
the implementation of request-response communications on top of
publish-subscribe interactions. Remarkably, the mapping that we present in this
work is general and could be used also in other contexts.

\end{itemize}

The flexibility provided by Jolie can be used to support and interconnect
multiple IoT islands, as discussed above. Jolie supports also more advanced
scenarios where the selection of the protocol to use changes according to
internal or environmental conditions, such as available energy or quality of
communication, but of course this requires some capability of switching the
protocol also from the side of the Things, which may not be the case in many
practical situations.

Indeed, in the rest of the paper, we omit to model IoT devices --- like Arduino
and other microcontrollers --- that are at the edge of the network, since they
are normally programmed by using low-level languages.  In principle, these
devices could be programmed by using Jolie-like languages, possibly extending
them to provide those low-level abstractions needed by programmers to access the
in-board sensors and actuators.  However, the constraints on the hardware and
the usually limited amount of energy available to edge devices require a
low-footprint, lightweight execution environment.  Although these requirements
could be achievable also for a language like Jolie, this would require a strong
engineering effort, which is not considered in this paper. We argue that this
direction of work is not urgent, since currently developers tend to program very
simple behaviors for edge devices~\cite{7123563}, which usually capture some
data (e.g., through one of their sensors) and then send them to other devices
(gateways, aggregators, servers). These other devices have more powerful
hardware and less constraints on energy consumption, and can then implement the
logic for the data processing. Hence, here we neglect the programming of edge
devices
%
%would require a considerable change in the actual patterns of interactions that
%are currently by developers. In fact,
%
and we focus on those devices that can both host the Jolie
runtime and whose topological context can benefit from the
flexibility offered by the language.
