\textit{Message Queue Telemetry Transport} (MQTT)~\cite{mqtt,mqtt-v3.1.1} is a
publish/subscribe messaging application protocol built on top of
the TCP transport protocol. 

A typical publish/subscribe interaction pattern can be diagrammatically
represented as in \cref{fig:pub_sub_encoding_one_way} where:

\begin{enumerate}
  \item a Subscriber subscribes to topic (a) at some Broker;
  \item a Publisher publishes a message to topic (a) at the same Broker;
  \item the Broker forwards the message to topic (a) to the Subscriber.
 \end{enumerate}
 
\begin{figure}[th]
% \hspace{1em}\resizebox{0.4\textwidth}{!}{
\begin{sequencediagram}
\setUmlSeqChartStyle

  \newinst{sub}{Subscriber}
  \newinst[1.8]{bro}{Broker}
  \newinst[1]{pub}{Publisher}

  \mess{sub}{{1) Subscribe to (a)}}{bro}
  \mess{pub}{{2) Publish at (a)}}{bro}
  \mess{bro}{{3) Forward msg in (a)}}{sub}
\end{sequencediagram}
% }
\caption{\label{fig:pub_sub_encoding_one_way}
Typical publish/subscribe interaction pattern.}
\end{figure}

On top of the basic mechanism of publish/subscribe, MQTT defines three levels of
quality of service (QoS) for the delivery of each message published by a
publisher. QoS levels determine whether messages can be lost and/or duplicated.
Concretely, QoS levels are as follows:
\begin{itemize}
  \item \textit{At most once} --- the message can be lost, no duplication
  can occur.
  \item \textit{At least once} --- delivery of the message is guaranteed, but
  duplication may occur.
  \item \textit{Exactly once} --- delivery of the message is guaranteed and
  duplication cannot occur.
\end{itemize}

To present how we model the MQTT protocol in Jolie, we first detail the simpler
case of \code{OneWay} communications in \cref{sub:ow_in_mqtt}. Then, we address
the more complex case of \code{RequestResponse}s in \cref{sub:rr_in_mqtt}. Both
mappings are general, i.e., in principle they can be applied to implement
one-way and request-response communication patterns on top of any
publish/subscribe protocol. Remarks on the implementation are provided in 
\cref{sub:impl_mqtt}.

\subsection{One-Ways in MQTT}
\label{sub:ow_in_mqtt}

We first consider the case of inbound communications and then the case of
outbound communications.

To exemplify \code{OneWay} inbound communications, we use the example in
\cref{lst:mqtt_simple_ex}, which is a revision of the example in
\cref{temperature_interface} by omitting the HTTP and CoAP ports and by adding an
MQTT \code{inputPort}.

\begin{lstlisting}[%
  basicstyle=\footnotesize\ttfamily,
  label=lst:mqtt_simple_ex, 
  caption={Code of the Collector Example, revised for MQTT.}]
interface TemperatureInterface {
  OneWay: receiveTemperature( string )
}

inputPort CollectorPort3 {
 Location: "socket://localhost:8050"
 Protocol: mqtt {
  .broker = "socket://iot.eclipse.org:1883"
 }
 Interfaces: TemperatureInterface
}

main {
  receiveTemperature( data )
}
\end{lstlisting}

As expected, the program behavior and the structure of the \code{inputPort}
are unchanged. Main novelties are:
%
\begin{itemize}
  \item the used \code{Location} (line 6) has the prefix 
  \code{"socket://"} (as seen in the HTTP port) since MQTT relies on TCP
  transport protocol;
  \item the used \code{Protocol} (line 7) is \code{mqtt};
  \item the \code{.broker} protocol parameter (line 8), which is
  compulsory, specifies the address of the Broker.
\end{itemize}

While the syntax and the effect of the communication primitive from the point of
view of the programmer are the same as the ones in \cref{temperature_interface},
the actual message exchanges performed to obtain such an effect are different.

Beyond defining such message exchanges, we also need to decide how to identify
the topic on which the message exchange is performed.

Regarding the message exchanges, from the point of view of the programmer, an
inbound \code{OneWay} communication receives a datum from the communication
partner. To obtain the same effect using the publish/subscribe paradigm, one has
first to subscribe at the Broker to the chosen topic and then wait to receive a
message on that topic, forwarded by the Broker. How topics are selected will be
detailed later on. The execution of a reception on a \code{OneWay} operation
comprises two actual communications: a subscription from the program to the
Broker and a message delivery in the opposite direction. However, subscription
to topics and the execution of a message reception are logically separated and
can be done at different moments. Indeed, the subscription is performed when the
Jolie program is launched for all operations present in MQTT
\code{inputPort}s. This choice is more in line with the expected behavior of
Jolie programs --- and of Service-Oriented programs in general --- where
messages to operations, whose reception statements are not yet enabled, are
stored until the actual execution of the reception.
%
% Here, if the
% subscription is performed along with the execution of the
% \code{OneWay} operation, previous messages could be no more available.
%
In Jolie, the compulsory parameter \code{.broker} is needed precisely to know
the address at which the subscription needs to be performed. The address for the
delivery of the actual message is the usual \code{Location} of the
\code{inputPort}.

Regarding the selection of topics, similarly to what done for CoAP resources, in
MQTT we default to mapping Jolie operations to topics, otherwise we use the
\code{.osc} parameter \code{.alias} to enable loose coupling between operations
and topics. We remark that \code{.alias} in \code{inputPort}s have a different
behavior in MQTT with respect to HTTP and CoAP.\@ In CoAP the name of the resource extracted
from the received message is used to derive the correct instantiation of the
\code{.alias} template. The values resulting from the match are then
inserted among the elements of the payload before storing it in the target
variable \code{data}. Instead, in MQTT, the \code{.alias} parameter is used to
identify the topic for subscription. For example, in \cref{lst:mqtt_simple_ex},
one could add the \code{Protocol} parameter \code{.osc.receiveTemperature.alias
= "temperature"} to specify that the selected topic for operation
\code{receiveTemperature} is \code{"temperature"}. Note that, since
there is no outgoing data, templates in MQTT \code{inputPort}s, such as
\code{"temperature"} in the example, are constants (we require all such
constants defined within the same
\code{inputPort} to be distinct). Having only constant aliases is not a relevant
limitation in the context of IoT, where topics are mostly statically fixed.
Addressing this limitation without disrupting the uniformity of the Jolie
programming model is not trivial and is left as future work.

% \begin{figure}[ht]
% \begin{sequencediagram}
% \setUmlSeqChartStyle

%   \newinst{sub}{Collector}
%   \newinst[2]{bro}{Broker}
%   \newinst[1]{pub}{Device}

%   \mess{sub}{{1) Subscribe to (receiveTemperature)}}{bro}
%   \mess{pub}{{2) Publish on (receiveTemperature)}}{bro}
%   \mess{bro}{{3) Forward msg on (receiveTemperature)}}{sub}
% \end{sequencediagram}
% \caption{\label{fig:mqtt_simple_ex}
% Representation of the example in \cref{lst:mqtt_simple_ex}.}
% \end{figure}

To conclude the mapping of \code{OneWay} operations in MQTT, we consider here
the case of outbound operations, exemplified in \cref{lst:oneway_out}.
%
Outgoing \code{OneWay} operations simply cause the publication of the value
passed as the parameter of the invocation (line 17) at the Broker. The
address of the Broker is defined by the \code{Location} (line 6) of the
\code{outputPort} \code{Broker}. The topic is derived from the name of the
\code{operation} and the parameter of the invocation, using protocol parameter
\code{.alias} as usual. Being an MQTT publication, we specify the
\code{.QoS} protocol parameter (line 10), which selects the QoS level
``Exactly once'' for the operation \code{setTmp}.

\begin{lstlisting}[basicstyle=\footnotesize\ttfamily,caption={Example of
outgoing MQTT \code{OneWay} communication.},label=lst:oneway_out]
interface ThermostatInterface {
 OneWay: setTmp( TmpType )
}

outputPort Broker {
 Location: "socket://iot.eclipse.org:1883"
 Protocol: mqtt {
  .osc.setTmp << {
   .format = "raw",
   .QoS = 2, // exactly once QoS
   .alias = "%!{id}/setTemperature"
 }
}
Interfaces: ThermostatInterface }

main { 
 setTmp@Broker( 24 { .id = "42" } )
}
\end{lstlisting}

\subsection{Request-Responses in MQTT}
\label{sub:rr_in_mqtt}

To discuss \code{RequestResponse} communications, let us consider the
example in \cref{lst:coap_example}, revised in \cref{lst:mqtt_example}
by replacing the CoAP protocol with MQTT. We omit \code{OneWay}
communications and concentrate on the
outbound \code{RequestResponse}. Afterwards, we will also discuss the
dual inbound \code{RequestResponse}.

\begin{lstlisting}[
basicstyle=\footnotesize\ttfamily,
label=lst:mqtt_example,
caption=Jolie controller communicating over MQTT.]
interface ThermostatInterface {
  RequestResponse: getTmp( TmpType )( int )
}

outputPort Broker {
 Location: "socket://iot.eclipse.org:1883"
 Protocol: mqtt {
  .osc.getTmp << {
   .format = "raw",
   .QoS = 2, // exactly once QoS
   .alias = "%!{id}/getTemperature",
   .aliasResponse = "%!{id}/getTempReply",
  }
 }
 Interfaces: ThermostatInterface
}

main {
 getTmp@Broker( { .id = "42" } )( temp )
}
\end{lstlisting}

Syntactically, the main novelty with respect to the \code{outputPort} in
\cref{lst:oneway_out} is the addition of \code{Protocol} parameter
\code{.aliasResponse}. This parameter specifies the name of the topic
where the receiver will publish its response.

From the point of view of the programmer, an outbound \code{RequestResponse} is
composed of an outgoing communication followed by an inbound reply. The outgoing
communication is implemented using the approach already seen for \code{OneWay}
communications, i.e., using the \code{.alias} \code{Protocol} parameter to
identify the topic. Then, one has the issue of relating the outgoing request
with its reply. Many standard point-to-point communication technologies, such as
HTTP/TCP and the already discussed CoAP/UDP, support request-response
communications by defining means to link a given outgoing request to its reply.
MQTT does not provide dedicated means to do such a linking, hence we exploit the
content and the topic of messages to this end.
%
We identify the topic for the reply with the
\code{.aliasResponse} \code{Protocol} parameter. Like for \code{.alias}
parameters, the template of the \code{.aliasResponse} parameter is instantiated
using the content of the message sent in the behavior. For example, in
\cref{lst:mqtt_example}, we use \code{.id} in line 19 to obtain
\code{"42/getTemperature"} and \code{"42/getTempReply"}, respectively the
publication and reply topics.

We can now describe the pattern of interactions that we use to implement the
outgoing \code{RequestResponse} communication at line 19 in
\cref{lst:mqtt_example}. As a reference, the pattern of interactions is depicted
in the left part of \cref{fig:pub_sub_RR}. We will describe the right part later
on, after having introduced inbound request-response communications.

\begin{figure}[ht]
% \hspace{1em}\resizebox{0.4\textwidth}{!}{
\begin{sequencediagram}
\setUmlSeqChartStyle

  \newinst{pro}{Controller}
  \newinst[1.4]{bro}{Broker}
  \newinst[1.4]{the}{Thermostat}

  \postlevel
  \mess{the}{{1) Subscribe to "42/getTemperature"}}{bro}
  \stepcounter{seqlevel}
  \mess{pro}{{2) Subscribe to "42/getTempReply"}}{bro}
  \postlevel
  \mess{pro}{{3) Publish to "42/getTemperature"}}{bro}
  \mess{bro}{{4) Forward msg in "42/getTemperature"}}{the}
  \postlevel
  \mess{the}{{5) Publish to "42/getTempReply"}}{bro}
  \mess{bro}{{6) Forward msg in "42/getTempReply"}}{pro}

\end{sequencediagram}
% }
\caption{\label{fig:pub_sub_RR} Interaction in the home automation
example in MQTT.}
\end{figure}

First, the controller subscribes to the reply topic 
\code{"42/getTempReply"} at the Broker. Then, the controller
sends to the Broker the request message on topic
\code{"42/getTemperature"}; the payload of the message contains the reply topic
\code{"42/getTempReply"}. The execution of the \code{RequestResponse}
terminates when the Broker forwards the reply received on topic
\code{"42/getTempReply"} to the controller.

Differently from inbound \code{OneWay} communications, here we do not subscribe
to the reply topic when the program is launched. Indeed, it would be useless
since no relevant message can arrive on this topic before the controller sends
its message to the Broker and, by anticipating the subscription, it would
complicate the usage of runtime information in templates.

To exemplify inbound \code{RequestResponse} communications, we assume that the
thermostat in our example is programmed in Jolie. We report its code in
\cref{lst:in_mqtt_example}.

\begin{lstlisting}[
basicstyle=\footnotesize\ttfamily,
label=lst:in_mqtt_example,
caption=Jolie thermostat communicating over MQTT.]
interface ThermostatInterface {
  RequestResponse: getTmp( TmpType )( TmpType )
}

inputPort Thermostat {
 Location: "socket://localhost:9000"
 Protocol: mqtt {
  .broker = "socket://iot.eclipse.org:1883";
  .osc.getTmp << {
   .format = "raw",
   .alias = "42/getTemperature"
  }
 }
 Interfaces: ThermostatInterface 
}

main {
 getTmp( temp )( temp ){
  // retrieves temperature and stores
  // it within the root of variable temp
 }
}
\end{lstlisting}

At line 11 in \cref{lst:in_mqtt_example}, the \code{.alias} parameter
\code{"42/getTemperature"} is static, as usual for \code{inputPort}s.

When the thermostat program is launched, it subscribes to topic
\code{"42/getTemperature"}. When a message on this topic arrives, the reply
topic, e.g., \code{"42/getTempReply"}, is extracted and the rest of the payload
(empty in this case) is passed to the behavior. The body of the
\code{RequestResponse} (lines 19--20) is executed to compute the return value.
Finally, the return value is published on the reply topic
\code{"42/getTempReply"}.

We now summarize the exchange between the controller and the thermostat (left
part of \cref{fig:pub_sub_RR}):

\begin{enumerate}
  \item when the thermostat is started, it subscribes to topic \code{"42/getTemperature"} at the
  Broker;

  \item when the outgoing \code{RequestResponse} is executed, the controller
  subscribes to topic \code{"42/getTempReply"} at the Broker;

  \item the controller publishes the request message to topic
  \code{"42/getTemperature"};

  \item the Broker forwards the message in topic \code{"42/getTemperature"} to
  the thermostat;

  \item the thermostat computes the response and publishes it at topic 
  \code{"42/getTempReply"};

  \item the Broker forwards the message in topic \code{"42/getTempReply"} to the
  controller.

\end{enumerate}

We remark that \code{RequestResponse} operations are meant to be one-to-one
communications. To ensure this in a publish/subscribe setting while using the
approach above, one has to ensure that no other participant subscribes to the
selected topics, which essentially act as namespaces.

\subsection{Implementation of MQTT in Jolie.} % (fold)
\label{sub:impl_mqtt}

The implementation of MQTT in Jolie required the creation of two main classes:
the actual MQTT protocol and a generic publish/subscribe meta-channel that
bridges between the end-to-end style of Jolie communications and
publish/subscribe interactions. Furthermore, we had to update the Jolie class
for channel creation so that it could choose between the standard end-to-end
media and the new publish/subscribe meta-channel.

The MQTT protocol class both encodes and decodes messages and implements the QoS
policies of the MQTT standard. Concretely, as for CoAP, we based the
implementation of MQTT on Netty~\cite{maurer16}. The main difficulty in the
implementation of the protocol is the definition of the message patterns needed
to implement \code{OneWay} and \code{RequestResponse} communications, which have
been described above. Beyond being invoked when operations are executed, the
MQTT class is also invoked when the program is started, to perform port
initialization. In particular, this is when subscriptions to topics identified
in \code{inputPort}s are performed (along with the related connections to the
Brokers).

In addition to the MQTT protocol, we also implemented a generic
publish/subscribe meta-channel. Indeed, since Jolie is based on an end-to-end
communication pattern, it assumes that the caller requires the creation of a
connection to the server, which waits for inbound requests. For this reason,
given a certain medium, \code{inputPort}s and \code{outputPort}s
use a medium-specific implementation of, respectively,
a listener class and a sender class.

%
This pattern, separating listeners from senders, does not apply to
publish/subscribe protocols, where both the subscriber and the publisher need to
establish a connection with the broker. For this reason we use the
publish/subscribe meta-channel to bridge between the two styles. On the one
hand, it implements the interfaces of listeners and senders as required by
Jolie. On the other hand, it relies on the pre-existing Jolie sender classes
(TCP socket in the case of MQTT) to create the connection to the broker.