In the literature there are many proposals for platforms, middlewares, smart
gateways, and general systems, all aimed at solving the interoperability
problem arising from the current ``babel'' of IoT technologies (protocols,
formats, and languages). Without any claim of being complete, here we mention a
few notable examples which are somehow related to our current research.

Recently the W3C started the Web of Things (WoT) Working Group~\cite{w3c17}. The
aim of WoT is to define a standard stack of layered technologies, as well as
software architectural styles and programming patterns, to uniform and simplify
the creation of IoT applications. In this context, the W3C is working on a WoT
Architecture~\cite{wot.arch}. The main concept of the architecture is the notion
of ``servient'', a virtual entity that represents a physical IoT device.
Servients provide technology-independent, standard APIs that developers can use
to transparently operate in heterogeneous environments. Remarkably, both the
WoT proposal and ours concern high-level
abstractions for low-level access to devices provided via, e.g., HTTP, CoAP, and
MQTT.\@ However, while we propose a dedicated language, they provide API
specifications.
%
More in general, there are many proposals for the integration of WoT and IoT.\@
For example~\cite{dominique2011web} and~\cite{corredor2014lightweight} define
general platforms covering different layers of IoT, including an accessibility
layer which integrates concepts like smart gateways and proxies to facilitate
the connection of (smart) Things into the Internet infrastructure, using
architectural principles based on REST.
%
Smart gateways and proxies are used in several industrial proposals to
facilitate the development of applications. Common denominator of some of these
proposals, e.g.,~\cite{60, 61, 62}, is the abstraction of low-level
functionalities provided by embedded devices (e.g., connectivity and
communication over low-level protocols like ZigBee, Z-Wave, Wi/IP/UPnP, etc.).
Smart gateways are used also to translate (or integrate) CoAP into
HTTP~\cite{7811451,s150101217,7037719} and to integrate both CoAP and MQTT by
means of specific middlewares~\cite{6827678}. Eclipse IoT~\cite{EplicseIoT} is
an IoT integration framework proposed by the Eclipse IoT Working Group. Aim of
Eclipse IoT is to build an open IoT stack for Java, including the support for
device-to-device and device-to-server protocols, as well as the provision of
protocols, frameworks, and services for device management. There exist several
 European projects, notably INTER-IoT~\cite{7471373} and
symbIoTe~\cite{Gojmerac16}, that address the issue of interoperability in IoT
and have produced several concrete proposals. Finally, a work close to ours
is~\cite{Zhiliang11}, where a middleware converts IoT heterogeneous networks
into a single homogeneous network.

Although related to our aim in this paper, the cited proposals tackle the
problem of IoT integration from a framework perspective: they provide chains of
tools, each addressing a specific level of the integration stack. Differently,
we extend a language specifically tailored for system integration and advanced
flow manipulation, Jolie, to support integration of IoT devices. This offers a
single linguistic domain to seamlessly integrate disparate low-level IoT devices
and intermediate nodes (collectors, aggregators, gateways). Moreover, Jolie is
already successfully used for building Cloud-based, microservice
solutions~\cite{GabbrielliGGMM16,MelisPGC17}. This makes the language useful
also for assembling advanced architectures for IoT, e.g., to handle real-time
streaming and processing of data from many devices. The benefit, here, is that,
while solutions based on frameworks require dedicated proficiencies on each of
the included tools, Jolie programmers can directly work at any level of the IoT
stack, without the need to acquire specific knowledge on the tools in a given
framework.

To conclude our revision of related work, we narrow our focus on language-based
integration solutions for IoT. The work most related to ours is
SensorML~\cite{SensorML}.
%
SensorML, abbreviation of Sensor Model Language, is a modeling
language for the description of sensors and, more in general, of measurement
processes. Some features modeled by the language are: discovery and
geolocalization of sensors, processing of sensor observations, and
functionalities to program sensors and to subscribe to sensor events.
%
While some traits of SensorML are common to our proposal, the scopes of the two
languages sensibly differ. Indeed, while Jolie is a high-level language for
programming generic architectures (spanning from cloud-based microservices to
low-level IoT integrators), SensorML just models IoT devices, their discovery,
and the processing of sensor observations.

%
