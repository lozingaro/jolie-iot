In this section we illustrate, via an example, the main elements of
Jolie programs, highlighting their peculiarities for programming
distributed systems.

%As such, Jolie provides
%some basic constructs which are useful in the IoT context even though,
%as we discuss later, we need to add to the language some further  features in
%order to obtain a real IoT programming framework.

As already said, in Jolie, programs are divided into two
parts: deployment and behavior.

The deployment part of a Jolie program describes which \emph{ports} it
uses for communicating with other programs. More
precisely, \emph{input ports} describe how messages can reach a Jolie
program, while \emph{output ports} describe how messages are sent from
a Jolie program. Each port specifies the medium used for communication
(with the keyword
\code|Location|), the data protocol used (with the keyword
\code|Protocol|), and the supported interface (with the keyword
\code|Interfaces|). 

Currently, Jolie supports TCP/IP sockets, Bluetooth L2CAP, Java RMI,
and Unix local sockets as \code{Location}, and HTTP, JSON, XML,
SOAP, SODEP (Jolie binary protocol) and their respective SSL versions
as \code{Protocol}.

Interfaces are used to enforce compliance among the nodes of a
distributed application. Intuitively, in interfaces developers specify
the names of the available functionalities and what data structures
they accept and expect to send. In this way, just by having the
interface of a server, one knows which functionalities the server
provides, which data structures it accepts and which data structures
it sends back. In Service-Oriented Programming functionalities are
called operations and can be grouped in two main
classes: \emph{One-way} operations, which allow one to receive a
message\footnote{The dual operation of asynchronously sending a
message is called \emph{notification}.},
and \emph{Request-Response} operations, which allow one to receive a
message and send back an answer\footnote{The dual operation of sending a
message and waiting for an answer is called
\emph{solicitResponse}.}.

As an illustrative example, we consider a classic interaction between
some edge devices and a collector node: the devices sense some
temperature and send it to the collector.

In Jolie, the behavior of the collector is implemented by the code below
%
\begin{lstlisting}
main {
  getTemperature( data );
  ...
}
\end{lstlisting}
%
Following the above terminology, \code{getTemperature} is a
one-way input operation, i.e., an input label that a device can use to
call the collector, in this case to provide a temperature measurement.
Here, one-way means that the communication is asynchronous, i.e., the
device can continue without waiting for the collector to actually
consume the delivered message. When consumed, the content of the
message is assigned to variable \code{data}. The semicolon
(\code{;}) indicates sequence, i.e., that the code before the
semicolon needs to complete before the code following it could start.

The behaviour above needs to be complemented by a deployment,
detailing how the collector communicates with other devices.
Such an information is defined within \emph{ports}, declarative
constructs where programmers can specify:
%(the list is not comprehensive):
%
\begin{description}
  \item[Location:] the communication medium (e.g., TCP/IP, Bluetooth)
  and address used to contact the device;
  \item[Protocol:] the data format (e.g., HTTP, SOAP) and other optional settings, e.g., protocol-specific parameters;
  \item[Interfaces:] a set of interfaces, each of which is a collection
  of operation labels associated with the types of the exchanged messages. Each
  operation can either be a \code{OneWay} or a
  \code{RequestResponse}.
%  , the latter being a synchronous exchange, where
%  the invoker sends a message and waits for a response from the addressee node.
\end{description}

In the current example the collector contains an
\code{inputPort} --- a port that defines an inbound
connection --- called \code{CollectorPort1}. The port specifies that the
collector accepts TCP/IP connections at URL
\code{"socket://collector.net:8000"} on the \code{HTTP} data protocol.
We also define the interface of the collector, called
\code{TemperatureInterface}, with construct
\code{interface}. Below we define the type of the message on operation
\code{getTemperature} to be a \code{string}.

\begin{lstlisting}[xleftmargin=1em]
interface TemperatureInterface {
  OneWay: getTemperature( string )
}

inputPort CollectorPort1 {
  Location: "socket://collector.net:8000"
  Protocol: HTTP
  Interfaces: TemperatureInterface
}
\end{lstlisting}

Interfaces are specifically thought to be shared between nodes to enforce
compliance. In Jolie, both the sender and the receiver check whether ports
support an invoked operation and if the content of the related message complies
with the expected type defined in the interface. In this way, developers can
formally agree on the interfaces of the nodes within their network, which is a
prerequisite for and gives support to the advocated practice of API-driven
development~\cite{Schon2017}.

We conclude this section by considering an interesting case where Jolie would
show its strengths: after an update of the IoT network in the example, the
collector needs to be able to receive temperatures from both the pre-existing
devices (which worked on TCP/IP over HTTP) and the new devices that rely on the
CoAP protocol -- a renowned data protocol for IoT~\cite{coap}. As standard for
CoAP, the devices communicate over the unreliable communication protocol UDP.

Assuming the integration of CoAP and UDP in Jolie, which is one of the
main contributions of the present paper, to let the collector interact
with the new devices, it is sufficient to define an
additional \code{inputPort} that supports
\code{TemperatureInterface} and whose Location and Protocol are
respectively an UDP location and the CoAP protocol. Concretely, besides port
\code{Collector} we also define port \code{Collector2}.  Note that the
behaviour of the program (specified in \code{main}) remains the
same. For completeness, we report the whole code of the collector
below.
%

\begin{lstlisting}[xleftmargin=1em,label=temperature_interface]
interface TemperatureInterface {
  OneWay: getTemperature( string )
}

inputPort CollectorPort1 {
  Location: "socket://collector.net:8000"
  Protocol: HTTP
  Interfaces: TemperatureInterface
}

inputPort CollectorPort2 {
  Location: "datagram://coap.me:5683"
  Protocol: CoAP
  Interfaces: TemperatureInterface
}

main {
  getTemperature( data );
  ...
}
\end{lstlisting}
In the above listings is shown the Temperature Interface exposed both via HTTP and via CoAP.
