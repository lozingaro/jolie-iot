As mentioned above, Jolie currently supports some of the main technologies used
in SOAs. However, only a limited amount of IoT devices uses the media and
protocols already supported by Jolie. Indeed, protocols such as
CoAP/REST~\cite{doi:10.17487/RFC7252,coap} and MQTT~\cite{mqtt-v3.1.1,mqtt},
which are widely used in IoT scenarios, are not yet implemented in Jolie.
Implementing these protocols is essential in order to allow Jolie programs to
directly interact with the majority of IoT devices. However there are some
challenges linked to the implementation of these technologies within Jolie:

\begin{itemize}
  \item \textit{lossless vs.\@ lossy protocols} --- In SOAs,
  machine-to-machine communication relies on lossless protocols, as there are no
  strict constraints on energy consumption or bandwidth, hence the number of
  message exchanges at the transport level needed for ensuring a message
  delivery is not critical. On the contrary, in IoT networks these
  constraints exist and are important, and the choice of the protocol needs to
  take them into account. Many protocols, and the CoAP application protocol in
  particular, rely on the UDP transport protocol --- a connectionless protocol
  that gives no guarantee on the delivery of messages, but allows one to limit
  message exchanges and energy and bandwidth consumption. Since Jolie assumes
  lossless communications, the inclusion of connectionless protocols in the
  language requires careful handling to prevent misbehaviors;

  \item \textit{point-to-point vs.\@ publish-subscribe} --- In order
  to provide language constructs that do not depend on the chosen protocol, we
  need to find a uniform setting covering both point-to-point communications,
  such as the ones of HTTP and CoAP, and publish-subscribe communications
  typical of MQTT.\@ Jolie already provides language constructs usable with many
  communication protocols, hence the less disruptive approach is to use the same
  constructs, which are designed for a point-to-point setting, also for MQTT.\@
  This requires to find for each point-to-point construct a corresponding effect
  in the publish-subscribe paradigm, such that typical programming patterns
  produce similar effects in both settings. In this way, one can program a
  unique behavior valid for both point-to-point and publish-subscribe
  scenarios.
  
\end{itemize} 

We detail how we integrate CoAP/UDP and MQTT in the Jolie language respectively
in \cref{sub:coap,sub:mqtt}.
%
The Jolie language interpreter, including our extensions at version 1.0, is
available at~\cite{jiot}. The integration of our extension into an official
Jolie release is ongoing work.
